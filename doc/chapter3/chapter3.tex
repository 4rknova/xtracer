\chapter{Θεμελιώδη γεωμετρικά σχήματα}

\begin{sloppypar}
\paragraph{}
	Στο προηγούμενο κεφάλαιο είδαμε πως ορίζονται οι αλληλεπιδράσεις του φωτός με τις διάφορες επιφάνειες, τις
βασικές αρχές του αλγορίθμου του ray tracing και τον τρόπο με τον οποίο δημιουργούνται οι ακτίνες, ξεκινώντας 
από τη κάμερα. Το ερωτημα που παραμένει αναπάντητο είναι πως γνωρίζουμε αν μια ακτίνα έχει χτυπήσει την επιφάνεια 
ενός αντικειμένου της σκηνής, σε κάποιο σημείο και πως μπορούμε να εντοπίσουμε αυτό το σημείο στο χώρο. Η απάντηση 
στο ερώτημα αυτό είναι ιδιαίτερα απλή. Το μόνο που χρειάζεται να γνωρίζουμε είναι η μαθηματική εξίσωση του εν λόγω 
αντικειμένου. Μπορούμε με ευκολία να αναπαραστήσουμε οποιοδήποτε αντικείμενο συνδυάζοντας τους τύπους της ευθείας και
του αντικειμένου που μας ενδιαφέρει. Παρακάτω θα ορίσουμε κάποια θεμελιώδη γεωμετρικά σχήματα που μπορούν να 
χρησιμοποιηθούν για την αναπαράσταση πιο πολύπλοκων δομών.

\paragraph{}
Αρχικά θα οριστεί η εξίσωση της ευθείας, που περιγράφει την ακτίνα φωτός.

\begin{equation}
x(t) = o + td
\end{equation}
\paragraph{}
όπου\\

\begin{equation}
o
\end{equation}
είναι το σημείο εκκίνησης της ακτίνας.

\begin{equation}
d
\end{equation}
είναι η διεύθυνση της.

\section{Σφαίρα}
\paragraph{}
	Το πιο συνηθισμένο αντικείμενο που συναντάμε σε ένα ray tracer είναι η σφαίρα, τα σημεία της οποίας ορίζονται από
την εξίσωση.

\begin{equation}
(x - c)^2 = r^2
\end{equation}
\paragraph{}
όπου\\

\begin{equation}
c
\end{equation}
\paragraph{}
είναι το κέντρο της σφαίρας

\begin{equation}
r
\end{equation}
\paragraph{}
είναι η ακτίνα της σφαίρας

\paragraph{}
	Αντικαθιστώντας τον τύπο της ευθείας στο x στο τύπο της σφαίρας και λύνοντας ως προς t καταλήγουμε σε μια διακρίνουσα. 
Αν η διακρίνουσα έχει δύο λύσεις τότε η ακτίνα τέμνει τη σφαίρα σε δύο σημεία από τα οποία μας ένδιαφέρει το κοντινότερο, 
δηλαδή η μικρότερη από αυτές τις δύο τιμές. Αν η διακρίνουσα είναι μηδέν τότε η ακτίνα είναι εφαπτόμενη στη σφαίρα και 
μπορούμε να αγνοήσουμε τη λύση. Τέλος, αν η διακρίνουσα είναι μικρότερη του μηδενός τότε η ακτίνα δεν τέμνει τη σφαίρα.


\section{Επίπεδο}
\paragraph{}
	Το επόμενο θεμελειώδες γεωμετρικό σχήμα που θα μας απασχολήσει είναι το άπειρο επίπεδο (infinite plane). Εργαζόμαστε
με την ίδια μέθοδο. Η εξίσωση του επιπέδου είναι:

\begin{equation}
(x(t) \cdot N) + d = 0
\end{equation}

\paragraph{}
όπου\\

\begin{equation}
N
\end{equation}
είναι το normal του επιπέδου.

\begin{equation}
d
\end{equation}
είναι η απόσταση του επιπέδου, από την αρχή των αξόνων.

\paragraph{}
Αντικαθιστώντας τον τύπο της ευθείας στο x στο τύπο του επιπέδου, καταλήγουμε στον τύπο:

\begin{equation}
t = \frac{- ( P_0 \cdot N ) + d}{V \cdot N}
\end{equation}
\paragraph{}
με\\

\begin{equation}
V \cdot N != 0
\end{equation}
\paragraph{}
Αν το t είναι μικρότερο του μηδενός, τότε το επίπεδο βρίσκεται πίσω από το σημείο εκκίνησης της ακτίνας και επομένως το σημείο 
απορρίπτεται.


\section{Τρίγωνο}
\paragraph{}
	Το τρίγωνο αποτελεί το βασικό δομικό συστατικό για πολλές πολύπλοκες δομές. Για την εύρεση της τομής με μια ακτίνα φωτός,
μετασχηματίζουμε το τρίγωνο σε ένα μοναδιαίο τρίγωνο στην αρχή των αξόνων, στο επίπεδο ΥΖ και την ακτίνα κατά μήκος του ΧΧ'
(Tomas Moller - Ben Trumbore). Στη συνέχεια αν βρεθεί σημείο τομής με το επίπεδο, υπολογίζουμε τις βαρυκεντρικές συντεταγμένες
του σημείου σε σχέση με το τρίγωνο και ελέγχουμε αν το άθροισμα τους είναι ίσο με τη μονάδα. Αν ναι τότε το σημείο βρίσκεται εντός
του τριγώνου.

\end{sloppypar}

% ------------------------------------------------------------------------
