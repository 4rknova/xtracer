\chapter{Εγχειρίδιο εφαρμογής}

\begin{sloppypar}

\section{Επιλογές προγράμματος}
\paragraph{}
	H βασική εφαρμογή έχει σχεδιαστεί για να χρησιμοποιείται απο τη γραμμή εντολών. Δέχεται ως είσοδο ένα αρχείο με
τη περιγραφή μιας σκηνής και έναν αριθμό από επιλογές που ορίζουν πως συμπεριφέρεται το λογισμικό. Παρακάτω δίνονται
όλες οι διαθέσιμες επιλογές για την έκδοση 0.6 μαζί με μια σύντομη περιγραφή της λειτουργίας τους.

\paragraph{}
	-res widthxheight
\paragraph{}
	Ορίζει το μέγεθος, σε εικονοστοιχεία της τελικής εικόνας.

\paragraph{}
	-threads n
\paragraph{}
	Ορίζει πόσα νήματα θα χρησιμοποιηθούν για τη δημιουργία της εικόνας. Εξ' ορισμού χρησιμοποιείται αριθμός νημάτων ίσος με
το πλήθος των επεξεργαστών που είναι ορατοί στο λειτουργικό σύστημα.

\paragraph{}
	-depth n
\paragraph{}
	Ορίζει το μέγιστο βάθος αναδρομής για τον υπολογισμό του φωτισμού.

\paragraph{}
	-dofsamples n
\paragraph{}
	Ορίζει το επίπεδο λεπτομέρειας, για τους υπολογισμούς που αφορούν το depth of field εφέ.

\paragraph{}
	-drv driver
\paragraph{}
	Ορίζει ποιος οδηγός θα χρησιμοποιηθεί για την παρουσίαση της τελικής εικόνας. Οι διαθέσιμοι οδηγοί είναι:\\
	- sdl: Παράθυρο SDL.\\
	- ppm: Εικόνα διαμόρφωσης PPM.\\
	- dum: Εικονικός οδηγός, για benchmarking.

\paragraph{}
	-gamma f
\paragraph{}
	Ορίζει το επίπεδο του gamma correction.

\paragraph{}
	-cam camera
\paragraph{}
	Ορίζει ποια κάμερα θέλουμε να χρησιμοποιήσουμε για να δημιουργήσουμε την εικόνα. Η δοθείσα κάμερα πρέπει να είναι
ορισμένη μέσα στη σκηνή.

\paragraph{}
	-lightpos
\paragraph{}
	Αντιμετωπίζει τις πηγές φωτός σαν κίτρινες σφαίρες. Χρησιμοποιείται για debugging μιας σκηνής.

\paragraph{}
	-antialiasing n
\paragraph{}
	Ορίζει το επίπεδο του antialiasing.

\paragraph{}
	-realtime
\paragraph{}
	Αν αυτό υποστηρίζεται απο τον επιλεγμένο driver, επιβάλει την ανανέωση της εξόδου κάθε φορά που υπολογίζονται νέα samples.

\paragraph{}
	-mod path:value
\paragraph{}
	Αφού φορτωθεί η σκηνή, αλλάζει τη τιμή του property που βρίσκεται στη διαδρομή path του δέντρου της σκηνής, δίνοντας του 
τη τιμή value.

\paragraph{}
	-v
\paragraph{}
	Αυξάνει το επίπεδο λεπτομέρειας των logs.

\paragraph{}
	-ver / -version
\paragraph{}
	Εμφανίζει λεπτομέρειες για την έκδοση του προγράμματος και τερματίζει.

\paragraph{}
	-help
\paragraph{}
	Εμφανίζει εν συντομία τον τρόπο χρήσης του προγράμματος και τερματίζει.

\section{Διαμόρφωση σκηνών}
	Για να ορίσουμε της σκηνές χρησιμοποιούμε μια domain specific γλώσσα. Η δομή ενός scene file έχει τη μορφή δένδρου.
Κάθε σύνθετη ιδιότητα ορίζεται σαν κόμβος στο δένδρο. Στο πρώτο επίπεδο έχουμε τα εξής βασικά nodes.

\paragraph{}
camera: Εδώ ορίζονται όλες οι διαθέσιμες κάμερες της σκηνής.
\paragraph{}
geometry: Εδώ ορίζεται όλη η γεωμετρία της σκηνής. Κάθε γεωμετρία δηλώνει τον τύπο της και ένα σετ από ιδιότητες.
\paragraph{}
light: Εδώ ορίζονται όλες οι φωτεινές πηγές.
\paragraph{}
material: Εδώ ορίζονται όλες οι υφές. Κάθε υφή δηλώνει τον τύπο της και ένα σετ απο ιδιότητες.
\paragraph{}
object: Εδώ ορίζονται όλα τα αντικείμενα που είναι ορατά στη σκηνή. Δεν αρκεί να έχουμε ορίσει τη γεωμετρία στο geometry
node. Για να μετέχει το αντικείμενο στη σκηνή πρέπει να οριστεί σε αυτό το node. Κάθε αντικείμενο χαρακτηρίζεται απο μια γεωμετρία και
μια υφή.

\paragraph{}
Eπίσης για κάθε σκηνή, στη κορυφή του δένδρου έχουμε τις ιδιότητες: name και description που περιέχουν το όνομα και μια σύντομη 
περιγραφή της σκηνής αντίστοιχα. Επιπροσθέτως ορίζονται και οι ιδιότητες ambient και ambient constant για την ψευδή διόρθωση των
μη απευθείας ανακλάσεων του φωτός, που δεν υπολογίζονται απο τη μέθοδο του ray tracing ( indirect illumination ). Δείγματα σκηνών
παρατίθενται στο παράρτημα Γ.

\end{sloppypar}

% ------------------------------------------------------------------------
