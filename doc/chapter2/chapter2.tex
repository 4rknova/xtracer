\chapter{Σκίαση}

\begin{sloppypar}

\section{Μοντέλα σκίασης}
\paragraph{}
	Για να προσομοιώσουμε διαφορετικά υλικά και υφές, χρησιμοποιούμε διάφορες συναρτήσεις που
περιγράφουν πως συμπεριφέρεται το φως όταν έρθει σε επαφή με μια επιφάνεια ύπο ορισμένες συνθήκες.
Αυτές οι συναρτήσεις έχουν κατά κύριο λόγο προκύψει εμπειρικά και επιχειρούν να περιγράψουν ένα
εύρος διαφορετικών υλικών η καθεμία. Παρακάτω περιγράφονται τα βασικότερα μοντέλα τα οποία και 
υλοποιούνται στο συνοδευτικό λογισμικό.

\subsection{Lambert}
\paragraph{}
	Αυτό το μοντέλο πήρε το όνομα του από τον σουηδό μαθηματικό Johann Heinrich Lambert. Είναι το πιο
απλοϊκό μοντέλο που συναντούμε και χρησιμοποιείται κυρίως στη περιγραφή των diffuse ανακλάσεων του φωτός
σε "σκληρές" επιφάνειες. Η φωτεινότητα μιας τέτοιας επιφάνειας, δεν επηρεάζεται από την γωνία θέασης, 
δηλαδή είναι ισοτροπική. Το εισερχόμενο φως ανακλάται ισόμορφα προς όλες τις κατευθύνσεις.
Παράδειγμα τέτοιων επιφανειών είναι το μη-φινιρισμένο ξύλο. Η συνάρτηση υπολογισμού είναι η εξής:

\begin{equation}
I_p =  (L \cdot N) I_l
\end{equation}

όπου\\

\begin{equation}
I_p
\end{equation}
είναι η τελική ένταση του φωτός.

\begin{equation}
L \cdot N
\end{equation}
είναι το dot γινόμενο του διανύσματος από το σημείο προς τη φωτεινή πηγή με το normal διάνυσμα της επιφάνειας στο σημείο.

\begin{equation}
I_l
\end{equation}
είναι η ένταση του φωτός της πηγής.

\subsection{Phong}
\paragraph{}
	Ο όρος phong shading αναφέρεται σε ένα σύνολο από τεχνικές που χρησιμοποιούνται στα γραφικά. Συμπεριλαμβάνει ένα μοντέλο
για τον υπολογισμό των ανακλάσεων του φωτός σε glossy επιφάνειες, καθώς και ένα τρόπο προσέγγισης του χρώματος σε ένα σημείο
μιας επιφάνειας μεσω της διαβάθμισης των normal κατά μήκος ενός πολυγώνου. Το τελευταίο μοντέλο χρησιμοποιείται συνήθως για την
δημιουργία πιο smooth επιφανειών σε αντικείμενα με χαμηλό αριθμό πολυγώνων. Στο συνοδευτικό λογισμiκό υλοποιούνται και οι δύο
τεχνικές, οι οποίες έχουν πάρει το όνομα τους από τον Bui Tuong Phong που τις δημοσίευσε το 1973 στα πλαίσια του Ph.D. στο 
πανεπιστήμιο της Utah. Την εποχή εκείνη, αυτές οι τεχνικές ήταν επαναστατικές για το πεδίο των γραφικών λόγω της απλότητας τους
και των σχετικά χαμηλών απαιτήσεων σε υπολογιστικό χρόνο για κάθε εικόνοστοιχείο. Η χρήση των τεχνικών σκίασης phong συνδυάζει 
τις diffuse και specular ανακλάσεις του φωτός και είναι ιδανική για τη περιγραφή glossy επιφανειών. Το μοντέλο αυτό προέκυψε από τη
παρατήρηση ότι οι γυαλιστερές επιφάνειες έχουν έντονα specular highlights συγκεντρωμένα σε ένα σημείο ενώ οι σκληρές επιφάνειες έχουν 
μεγαλύτερα specular highlights που αποδυναμώνονται σταδιακά. Η συνάρτηση υπολογισμού είναι η εξής:

\begin{equation}
I_p = k_a i_a + \sum\limits_{m \, \in \, lights} (k_d(\hat{L_m} \cdot \hat{N})i_{m,d} + k_s(\hat{R_m} \cdot \hat{V})^a i_{m,s} )
\end{equation}

όπου\\

\begin{equation}
I_p
\end{equation}
είναι η τελική ένταση του φωτός.

\begin{equation}
k_s
\end{equation}
είναι το ποσοστό ανάκλασης του specular παράγοντα.

\begin{equation}
k_d
\end{equation}
είναι το ποσοστό ανάκλασης του diffuse παράγοντα.

\begin{equation}
k_a
\end{equation}
είναι το ποσοστό ανάκλασης του ambient παράγοντα.

\begin{equation}
a
\end{equation}
είναι η shininess σταθερά για το material, με μεγαλύτερες τιμές για επιφάνειες που είναι πιο γυαλιστερές.

\begin{equation}
\hat{R_m} = 2(\hat{L_m} \cdot \hat{N}) \hat{N} - \hat{L_m}
\end{equation}
υπολογίζεται ως ανακλαση του διανύσματος με διεύθυνση από τη πηγή φωτός προς την επιφάνεια.

\subsection{Blinn-Phong}
\paragraph{}
	Το μοντέλο Blinn-Phong είναι μια τροποποίηση του phong μοντέλου και παράγει πιο ακριβή αποτελέσματα.
H βασική διαφορά είναι στο παράγοντα

\begin{equation}
\hat{R_m} \cdot \hat{V}
\end{equation}

ο οποίος αντικαθίσταται από το

\begin{equation}
\hat{N} \cdot \hat{H}
\end{equation}

με

\begin{equation}
H = \frac{L + V}{| L + V |}
\end{equation}

\section{Εφέ}
\paragraph{}
	Ο βασικός αλγόριθμος δεν είναι αρκετός για να προσομοιώσουμε όλες τις αλληλεπιδράσεις του
φωτός με τις διάφορες επιφάνειες. Έτσι για μεγαλύτερο βαθμό ρεαλισμού προσθέτουμε κάποιες επιπλέον
περιπτώσεις.

\subsection{Αντικατοπτρισμός}
\paragraph{}
	Ανάκλαση ονομάζεται το φαινόμενο της αλλαγής διεύθυνσης διάδοσης ενός μετώπου κύματος, μέσα στο ίδιο μέσο, 
από μια διαχωριστική επιφάνεια. Τα πιο συνηθισμένα παραδείγματα ανάκλασης είναι αυτά των κυμάτων φωτός, ήχου και νερού.
Οι επιφάνειες που προκαλούν το φαινόμενο της ανάκλασης ονομάζονται κάτοπτρα. Όταν έχουμε μια πολύ λεία επιφάνεια 
όπως ένας καθρέπτης, μπορούμε να δούμε τα είδωλα άλλων αντικειμένων να αποτυπώνονται πάνω της. Για να αναπαραστήσουμε 
αυτό το φαινόμενο, χρησιμοποιούμε αναδρομή. Όταν μια ακτίνα φωτός πέσει σε μια επιφάνεια με αυτή την ιδιότητα τότε 
δημιουργείται μια νέα ακτίνα με χρήση των κανόνων ανάκλασης.
\paragraph{}
1. H προσπίπτουσα ακτίνα, η ακτίνα ανάκλασης και το normal στο σημείο τομής της επιφάνειας, βρίσκονται στο ίδιο επίπεδο.
\paragraph{}
2. Η γωνία η οποία δημιουργείται ανάμεσα στη προσπίπτουσα ακτίνα και το normal ισούται με την γωνία ανάμεσα στο normal
και την ακτινα ανάκλασης.
\paragraph{}
3. Η προσπίπτουσα ακτίνα και η ακτίνα ανάκλασης βρίσκονται στις αντίθετες πλευρές του άξονα που ορίζεται από το normal.

\paragraph{}
H ακτίνα ανάκλασης υπολογίζεται ως εξής:

\begin{equation}
\hat{R} = \hat{I} - 2 ( \hat{I} \cdot \hat{N} ) \hat{N}
\end{equation}

Στη συνέχεια ακολουθούμε τη νέα ακτίνα εφαρμόζοντας τον αλγόριθμο του ray tracing αναδρομικά. Η ένταση που προκύπτει, 
πολλαπλασιάζεται με το ποσοστό ανάκλασης, το οποίο ορίζεται σα σταθερά στην υφή και προστίθεται στο αποτέλεσμα.

\subsection{Διάθλαση}
\paragraph{}
	Διάθλαση ονομάζεται το φυσικό φαινόμενο της εκτροπής της ευθύγραμμης τροχιάς διάδοσης που υφίστανται φωτεινά 
ή άλλα κύματα όταν διέρχονται από ένα διαπερατό από αυτά μέσον σε έτερο. Ιδιαίτερα, στην οπτική, Διάθλαση φωτός
χαρακτηρίζεται κάθε οπτικό φαινόμενο της εκτροπής της διεύθυνσης των φωτεινών ακτίνων κατά τη μετάβασή τους από ένα 
διαπερατό μέσο διάδοσης με δείκτη διάθλασης n1 σε άλλο μέσο διάδοσης με δείκτη διάθλασης n2, όπου n1 διάφορο του n2.
Το φαινόμενο αυτό, που οφείλεται στη διαφορετική ταχύτητα διάδοσης του φωτεινού κύματος και που εξαρτάται από το 
διαπερατό μέσο στο οποίο διαδίδεται το κύμα εξετάζει η Κυματική οπτική. Όταν φως διαδίδεται από ένα υλικό με μεγαλύτερο 
δείκτη διάθλασης (πυκνότερο υλικό) σε ένα με μικρότερο δείκτη, τότε κατά την ανάκλαση η φάση αντιστρέφεται κατά 180 μοίρες. 
Αντίθετα, όταν διαδίδεται από ένα αραιότερο σε ένα πυκνότερο υλικό, κατά την ανάκλαση η φάση του κύματος διατηρείται.

\paragraph{}
Ο νόμος του Snell, γνωστός και ως νόμος της διάθλασης περιγράφει τη σχέση ανάμεσα στη γωνία πρόσπτωσης και τη γωνία διάθλασης.
Ισχύει ότι:

\begin{equation}
n1 \sin\theta1 = n2 \sin\theta2
\end{equation}

\paragraph{}
Για μικρές γωνίες  είναι δυνατό να γίνει η προσέγγιση . Απο αυτή την προσέγγιση προκύπτουν και τα γεωμετρικά σφάλματα φακών.
Εκ των παραπάνω συνάγεται ότι στο "κενό" η πορεία των φωτεινών ακτίνων παραμένει αμετάβλητη, όταν δεν εκτρέπεται από βαρυτικά 
πεδία, όπως επίσης αμετάβλητη παραμένει κατά την διάδοσή τους μέσα σε ισόπυκνο διαπερατό μέσο π.χ. νερό, γυαλί κ.λπ.
Στην περίπτωση που φωτεινές ακτίνες διερχόμενες από ένα μέσον πέσουν κάθετα στην επιφάνεια του άλλου, τότε η γωνία πρόσπτωσης 
είναι μηδενική με αποτέλεσμα και η γωνία διάθλασης να είναι και αυτή μηδενική π.χ. ακτίνες φωτός από τον αέρα προσπίπτουσες 
κάθετα σε νερό συνεχίζουν στην ίδια διεύθυνση.

\paragraph{}
Η ακτίνα διάθλασης υπολογίζεται ως εξής:

\begin{equation}
T = \frac{n_1}{n_2} I + (\,\frac{n_1}{n_2} |\hat{I} \cdot \hat{N}| - \sqrt{1 - (\frac{n_1}{n_2})^2 (1 - (\hat{I} \cdot \hat{N})^2}\,) N
\end{equation}

\subsection{Βάθος πεδίου}
\paragraph{}
	Έως τώρα κάναμε χρήση ενός ιδιαίτερα απλοϊκού μοντέλου κάμερας, ( pinhole camera ). Εδώ θα δούμε μια επέκταση του μοντέλου
αυτού που μας επιτρέπει να εστιάσουμε σε αντικείμενα μέσα στη σκηνή. Το νέο μοντέλο που χρησιμοποιούμε όνομάζεται thin lense camera.
To βάθος πεδίου ορίζεται ως η απόσταση, ανάμεσα στο κοντινότερο και μακρινότερο σημείο στη σκηνή που παρουσιάζονται "έντονα". 
Ορίζουμε δύο νέες ιδιότητες στο υπάρχων μοντέλο. Πρώτα ορίζουμε το διάφραγμα του φακού, που είναι το μέγεθος της ωπής από την οποία
θα περάσουν οι ακτίνες του φωτός. Επιπροσθέτως, ορίζουμε το μήκος εστίασης που είναι την απόσταση του επιπέδου εστίασης. Η παραγωγή 
των ακτίνων πρώτου επιπέδου, σπάει σε δύο βήματα. Στο πρώτο βήμα δημιουργούμε τις ακτίνες με τον ίδιο ακριβώς τρόπο όπως και στη pinhole
κάμερα. Στο δεύτερο βήμα, βρίσκουμε το σημείο τομής με το επίπεδο προβολής και και το επίπεδο εστίασης. Στο σημείο τομής με το επίπεδο
προβολής δημιουργούμε μια νέα ακτίνα με τυχαία μετατόπιση εντός των ορίων του διαφράγματος και διεύθυνση προς το σημείο τομής στο επίπεδο
εστίασης. Τέλος όπως και στη pinhole κάμερα μετασχηματίζουμε την ακτίνα ώστε να προσανατολιστεί όπως και η κάμερα στη σκηνή.

\end{sloppypar}
