% Thesis Introduction 
\chapter{Εισαγωγή}

\ifpdf
    \graphicspath{{Introduction/IntroductionFigs/PNG/}{Introduction/IntroductionFigs/PDF/}{Introduction/IntroductionFigs/}}
\else
    \graphicspath{{Introduction/IntroductionFigs/EPS/}{Introduction/IntroductionFigs/}}
\fi

\begin{sloppypar}

\paragraph{}
Για να αναπαραστήσουμε και να προσομοιώσουμε ένα εικονικό περιβάλλον σε ένα υπολογιστικό σύστημα,
είναι απαραίτητο να βρεθεί μια μαθηματική περιγραφή των οντοτήτων που εμπεριέχονται σε αυτό, 
καθώς και των ιδιοτήτων τους. Για το σκοπό αυτό έχουν αναπτυχθεί πολλές τεχνικές,
καθε μια από τις οποίες έχει διαφορετικές εφαρμογές ανάλογα με τα χαρακτηριστικά της. Η πιο
διαδεδομένη από αυτές τις τεχνικές στη σφαίρα των real time απεικονίσεων, είναι η scanline 
rendering. Αυτό οφείλεται στο ότι μπορεί να δημιουργήσει αρκετά πειστικές εικόνες γρήγορα. 
Εν τούτοις υπάρχουν περιορισμοί στην ποιότητα του αποτελέσματος, κυρίως διότι δεν μοντελοποιούνται 
οι φυσικές ιδιότητες του πραγματικού κόσμου αλλά χρησιμοποιούνται αφαιρετικά προσεγγιστικά μοντέλα. 
Αυτό το κενό έρχεται να καλύψει εν μέρει η τεχνική του ray tracing.

\paragraph{}
Με τον όρο ray tracing αναφερόμαστε στην τεχνική η οποία χρησιμοποιείται για να δημιουργηθεί μία
εικόνα, ακολουθώντας την πορεία του φωτός καθώς αυτό περνάει μέσα από τα διακριτά εικονοστοιχεία 
του επιπέδου προβολής και κινείται μεσα σε μια σκηνή, αλλάζοντας πορεία επηρεαζόμενο από τις 
ιδιότητες που χαρακτηρίζουν κάθε επιφάνεια στην οποία αυτό προσπίπτει. Έτσι μπορούμε να έχουμε 
αντικείμενα με ημιδιαφανείς επιφάνειες, με διαφορετικές υφές και χρώματα, οι οποίες προκαλούν 
την αντανάκλαση ή την διάθλαση αλλαζοντας τη πορεία του φωτός. Για παράδειγμα, αν έχουμε τέσσερεις 
κρυστάλλινες σφαίρες, διαφορετικού χρώματος και μεγέθους, τη μία πολύ κοντά στην άλλη, θα πρέπει 
στην επιφάνεια της κάθε μπάλας να αποτυπώνονται οι υπόλοιπες τέσσερις, όχι μόνο το είδωλο τους, 
αλλά τόσο το χρώμα, όσο και το μέγεθός τους όπως θα γινόταν και σε ένα πραγματικό περιβάλλον. 
Παρακάτω θα παρουσιαστούν οι τεχνικές που χρησιμοποιούνται για να επιτύχουμε αυτό το αποτέλεσμα 
καθως και το πλήρες μαθηματικό υπόβαθρο πίσω από αυτές.

\end{sloppypar}
