% Thesis Abstract -----------------------------------------------------

%\begin{abstractslong}   %uncommenting this line, gives a different abstract heading
\begin{abstracts}        %this creates the heading for the abstract page

Η τεχνική του raytracing χρησιμοποιείται για την δημιουργία συνθετικών εικόνων ακολουθώντας
τη πορεία του φωτός καθώς αυτό περνάει απο το κάθε εικονοστοιχείο του επιπέδου προβολής.
Με αυτό το τρόπο μπορούμε με ευκολία να δημιουργήσουμε φωτορεαλιστικές απεικονίσεις ενός μαθηματικά
μοντελοποιημένου περιβάλλοντος, κάτι στο οποίο άλλες τεχνικές υστερούν. Η παρούσα μελέτη
έχει ως σκοπό την παρουσίαση των βασικών αλγορίθμων πίσω από τη τεχνική του ray tracing, καθώς
και τη διερεύνηση πιθανών υπολογιστικών βελτίστοποιήσεων με σκοπό την ελαχιστοποίηση του 
χρόνου και των πόρων που απαιτούνται για την σύνθεση της τελικής εικόνας. Επιπλέον παρουσιάζεται
το μαθηματικό πλαίσιο που ορίζει τις ιδιότητες των οντοτήτων και τα φυσικά μοντέλα τα οποία 
χρησιμοποιεί η εν λόγω τεχνική. Τέλος γίνεται υλοποίηση των παρουσιαζόμενων τεχνικών και
αλγορίθμων.

\end{abstracts}
%\end{abstractslong}

% ----------------------------------------------------------------------
