% Conclusions ----------------------------------------------------------

\def\baselinestretch{1}

\chapter{Συμπεράσματα}

\ifpdf
    \graphicspath{{Conclusions/ConclusionsFigs/PNG/}{Conclusions/ConclusionsFigs/PDF/}{Conclusions/ConclusionsFigs/}}
\else
    \graphicspath{{Conclusions/ConclusionsFigs/EPS/}{Conclusions/ConclusionsFigs/}}
\fi

\def\baselinestretch{1.66}

\begin{sloppypar}

\paragraph{}
	Έχοντας δει τους βασικούς αλγόριθμους της τεχνικής του ray tracing μπορούμε να καταλήξουμε σε
κάποια συμπεράσματα. Ο βασικός αλγόριθμός είναι απλός και υλοποιείται έυκολα. Επιπλέον μπορούμε να
προσομοιώσουμε με την ίδια ευκολία, φυσικά φαινόμενα που σε άλλες μεθόδους είναι δύσκολο έως και αδύνατο
να αναπαρασταθούν παρά μόνο με προσεγγιστικές μεθόδους. Τέτοια φαινόμενα είναι οι πολλαπλές ανακλάσεις του 
φωτός σε γυαλιστερές επιφάνειες, η διάθλαση του φωτός σε ημιδιαφανή αντικείμενα, το βάθος πεδίου και οι σκίες.
Ένα σημαντικό πλεονέκτημα της τεχνικής είναι η δυνατότητα να επεκταθεί με ευκολία σε μοντέλα καθολικής σκίασης
( global illumination ). Με μια τέτοια επέκταση μπορούμε να επιτύχουμε επιπλέον ρεαλισμό με πιο ακριβή σκίαση ( soft
shadows ), πολλαπλές ανακλάσεις για diffuse επιφάνειες, φωτισμό από area lights και ατμοσφαιρικά εφέ.

\paragraph{}
	Το μεγαλύτερο μειονέκτημα της μεθόδου είναι η ταχύτητα της, λόγω της μεγάλης υπολογιστικής ισχύς
που απαιτείται. Ανάλογα με την ανάλυση της τελικής εικόνας καθώς και τη πολυπλοκότητα της σκηνής,
μια εικόνα μπορεί να χρειαστεί απο λίγα λεπτά έως αρκετές μέρες για να δημιουργηθεί. Εντούτοις έχουν
αναπτυχθεί μέθοδοι και δομές για την επιτάγχυνση των υπολογισμών. Η πιο διαδεδομένη από αυτές τις μεθόδους
είναι η διαμέριση του χώρου ( space partitioning ). Με τη βοήθεια δομών δεδομένων που δημιουργούν ένα
δένδρο διαμέρισης της σκηνής μπορούμε να επιτύχουμε δραματική μείωση τον ελέγχων κάθε ακτίνας για πιθανές 
τομές με τα αντικείμενα της σκηνής. Αυτό βοηθάει ειδικότερα σε πολύπλοκα μοντέλα αποτελούμενα από μεγάλο 
αριθμό πολυγώνων. Για παράδειγμα σε ένα μοντέλο με 500.000 πολύγωνα, σε κάθε εικονοστοιχείο θα έπρεπε να 
ελέγχουμε όλα τα πολύγωνα με κάθε ακτίνα. Με τη χρήση ενός space partitioning σχήματος, οι έλεγχοι 
αυτοί μπορούν να μειωθούν στους 100 ή λιγότερους ανάλογα με τη γεωμετρία του αντικειμένου.

\end{sloppypar}

% ----------------------------------------------------------------------
