\chapter{Ο βασικός αλγόριθμος.}

\ifpdf
    \graphicspath{{Chapter1/Chapter1Figs/PNG/}{Chapter1/Chapter1Figs/PDF/}{Chapter1/Chapter1Figs/}}
\else
    \graphicspath{{Chapter1/Chapter1Figs/EPS/}{Chapter1/Chapter1Figs/}}
\fi

\section{Γενικά για τη μέθοδο}

\begin{sloppypar}

\paragraph{}
	Το ray tracing είναι μια τεχνική που χρησιμοποιείται στο πεδίο των γραφικών για τη
δημιουργία συνθετικών απεικονίσεων ένος μαθηματικά μοντελοποιημένου περιβάλλοντος. Η μέθοδος
αυτή προσπαθεί να εξομοιώσει τις ιδιότητες του φωτός και την συμπεριφορά του καθώς αυτό προσπίπτει
στις διάφορες επιφάνειες ενός χώρου. Είναι εύκολο κανείς να αντιληφθεί ότι μια τέτοια εξομοίωση θα
απαιτούσε τον αυστηρό ορισμό όλων των φυσικών ιδιοτήτων του φωτός και των οντοτήτων που αλληλεπιδρούν
με αυτό καθώς και μεγάλη υπολογιστική δύναμη. Κάτι τέτοιο αν και θα οδηγούσε σε απεικονίσεις που 
χαρακτηρίζονται απο εξαιρετικά υψηλό βαθμό ρεαλισμού, εν τούτοις παρουσιάζει μεγάλες δυσκολίες που 
καθιστούν την μέθοδο πρακτικά μη υλοποιήσιμη και τουλάχιστον με την παρούσα τεχνολογία, 
υπολογιστικά αδύνατη.

\paragraph{}
	Το ερώτημα λοιπόν που προκύπτει αμέσως είναι, με ποιον τρόπο μπορούμε να απλοποιήσουμε το φυσικό
μοντέλο που περιγράφει της αλληλεπιδράσεις αυτές ώστε να κάνουμε μια τέτοια εξομοίωση υπολογιστικά 
εφικτή. Η απάντηση είναι απλή. Λειτουργούμε αφαιρετικά με σκοπό την προσέγγιση ενός ρεαλιστικού
αποτελέσματος. Όπως θα δούμε και στη συνέχεια, στην μέθοδο του ray tracing κάνουμε αρκετές παραδοχές 
με σκοπό να ελαχιστοποιηθεί το υπολογιστικό κόστος.

\end{sloppypar}
 
	

Here is an equation\footnote{the notation is explained in the nomenclature section :-)}:
\begin{eqnarray}
CIF: \hspace*{5mm}F_0^j(a) &=& \frac{1}{2\pi \iota} \oint_{\gamma} \frac{F_0^j(z)}{z - a} dz
\end{eqnarray}

\section{Σκίαση}
and here I write more ...\cite{texbook}

\subsection{Aνακλάσεις}
... and some more ...

Now I would like to cite the following: \cite{latex} and \cite{texbook}
%and \cite{Rud73}.

I would also like to include a picture ...

\begin{figure}[!htbp]
  \begin{center}
    \leavevmode
    \ifpdf
%      \includegraphics[height=6in]{aflow}
    \else
 %     \includegraphics[bb = 92 86 545 742, height=6in]{aflow}
    \fi
    \caption{Airfoil Picture}
    \label{FigAir}
  \end{center}
\end{figure}

% above code has been macro-fied in Classes/MacroFile.tex file
%\InsertFig{\IncludeGraphicsH{aflow}{6in}{92 86 545 742}}{Airfoil Picture}{FigAir}

So as we have now labelled it we can reference it, like so (\ref{FigAir}) and it
is on Page \pageref{FigAir}. And as we can see, it is a very nice picture and we
can talk about it all we want and when we are tired we can move on to the next
chapter ...

I would also like to add an extra bookmark in acroread like so ...
\ifpdf
  \pdfbookmark[2]{bookmark text is here}{And this is what I want bookmarked}
\fi
% ------------------------------------------------------------------------


%%% Local Variables: 
%%% mode: latex
%%% TeX-master: "../thesis"
%%% End: 
